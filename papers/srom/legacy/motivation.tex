The intersection of additive processes and design optimization has introduced revolutionary capabilities for design, product development, and manufacturing. ``Complexity is free'' has been a common mantra with additive manufacturing processes \cite{jared2017additive}. However, anyone involved in the qualification or certification of additive processes or materials will acknowledge that complexity is currently not free due to the prevailing lack of understanding of advanced manufacturing processes \cite{jared2017additive}. One approach to address the stochastic nature of additive processes is to improve process determinism. An alternate, complementary approach is to account for these inherent uncertainties early in the design process by providing designers with uncertainty aware computational design tools that generate solutions that insure performance requirements are met and margins are quantified. \\

Uncertainties abound in any design scenario and include sources from requirements, boundary conditions, and environments; not just process capabilities, feedstocks, and final material properties. Advanced uncertainty quantification and propagation methodologies have been available for years \cite{ghanem2003stochastic,babuska2004galerkin,babuvska2007stochastic}, but even the most basic capabilities for design under uncertainty \cite{frangopol1995reliability,maute2003reliability,kharmanda2004reliability} remain unavailable to end users. Since existing tools do not account for uncertainties during analyses, there is no guarantee that their solutions are robust to these sources of uncertainty. While holding great potential and value for product performance and qualification, design under uncertainty is a significant challenge due to the computational resources necessary to create high-fidelity solutions. This computational toll limits the design iterations available to explore solutions robust to uncertainties. Therefore, to make design under uncertainty an integral part of the design process, critical algorithmic issues must be solved. First, novel sampling algorithms are needed to reduce sample sizes required to accurately quantify and propagate multiple sources of uncertainty. Second, algorithms must efficiently utilize all available computing resources to increase performance, speed, and accuracy. Third, these novel algorithms should be integrated into a reliable computational design tool accessible to end users. \\

A relatively untapped benefit of additive manufacturing is its potential to control material at the voxel level. This feature expands the design space available to designers and enables the fabrication of multi-functional parts that are not possible to create with conventional manufacturing processes today. Multi-material 3D printing is only a budding technology, but will certainly lead to increasingly functional designs. Gaynor \textit{et al.} \cite{gaynor2014multiple} realized compliant mechanism designs based on three-phase \cite{sigmund1997design,sigmund2001designPartOne,sigmund2001designPartTwo} topology optimization using the PolyJet additive manufacturing technology, which can print bulk materials covering a wide range of elastic moduli \cite{stratasys}. Even using single material printers, functional designs have been fabricated by varying the microstructure throughout the print to achieve varying elastic properties \cite{schumacher2015microstructures}. Although the ability to control material properties at the voxel level is attractive, it also presents risk and uncertainty since defects can subsequently be introduced at similar scales. Thus, novel uncertainty aware synthesis optimization tools are needed to aid designers explore this new multi-material design space and insure designs against the inherent imperfections that multi-material 3D printers can facilitate. \\

This work aims to apply SROMs to account for the uncertainty in the orientation of an applied load and elastic modulus of the compliant material while solving a multi-material structural topology optimization problem under uncertainty. An SROM is a low-dimensional, discrete approximation to a continuous random element comprised of a finite and usually small number of samples with varying probabilities. This non-intrusive approach enables efficient stochastic computations in terms of only a small set of samples and probabilities. The SROM concept was originally proposed in \cite{grigoriu2009reduced} and then further refined in \cite{warner2013stochastic}. The SROM approach has been demonstrated in multiple applications, including the determination of effective conductivities for random microstructures \cite{grigoriu2010effective}, the estimation of linear dynamic system states \cite{grigoriu2010linear,grigoriu2013solution}, inverse problems under uncertainty \cite{warner2015stochastic}, the quantification of uncertainty in intergranular corrosion rates \cite{sarkar2014stochastic}, and the prediction of the structural reliability of components containing laser welds \cite{emery2015predicting}. The primary strengths of SROMs are their ability to represent an underlying random quantity with low-dimensionality and to subsequently solve uncertainty propagation problems in a fraction of the computational time required by Monte Carlo methods.\\

This work, to the best of our knowledge, represents the first application of SROMs to multi-material structural topology optimization under uncertainty. The SROM framework represents a practical approach with the following strengths shown in this work: 1) it relies entirely on calls to existing deterministic solvers and optimization libraries, 2) it is easily parallelized and scalable, and 3) it is not specific to normally distributed random quantities. Additionally, SROMs give higher weight to important areas of the probability space \cite{warner2015stochastic}. This property yields low-dimensional approximations and thus relatively few calls to deterministic models. 
